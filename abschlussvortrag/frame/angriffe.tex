\subsection{ARP-Spoofing}
\begin{frame}{ARP-Spoofing Ablauf}
	\begin{columns}
		\begin{column}{0.5\textwidth}
			\begin{center}
				\only<1>{ARP-Anfrage Broadcast}
				\only<2>{ARP-Antwort Unicast}
				\only<3>{ARP-Spoofed-Antwort Unicast}
				\includestandalone[keepaspectratio, width= 0.9\textwidth]{tikz/baArpExample}
			\end{center}
		\end{column}
		\begin{column}{0.5\textwidth}
			\begin{center}
				\begin{onlyenv}<1>
					Anfrage (alice$\Rightarrow$alle)
					\begin{tabular}{|c|c|}
						\hline
						Feld & Wert \\
						\hline
						SHA & 00-00-00-00-00-01 \\
						\hline
						SPA & 1.1.1.1 \\
						\hline
						THA & FF-FF-FF-FF-FF-FF \\
						\hline
						TPA & 1.1.1.2 \\
						\hline
					\end{tabular}
				\end{onlyenv}
				\begin{onlyenv}<2>
					Antwort (bob$\Rightarrow$alice)
					\begin{tabular}{|c|c|}
						\hline
						Feld & Wert \\
						\hline
						SHA & \textcolor{ForestGreen}{00-00-00-00-00-02} \\
						\hline
						SPA & 1.1.1.2 \\
						\hline
						THA & 00-00-00-00-00-01 \\
						\hline
						TPA & 1.1.1.1 \\
						\hline
					\end{tabular}
				\end{onlyenv}
				\begin{onlyenv}<3>
					Antwort (eve$\Rightarrow$alice)
					\begin{tabular}{|c|c|}
						\hline
						Feld & Wert \\
						\hline
						SHA & \textcolor{red}{00-00-00-00-00-03} \\
						\hline
						SPA & 1.1.1.2 \\
						\hline
						THA & 00-00-00-00-00-01 \\
						\hline
						TPA & 1.1.1.1 \\
						\hline
					\end{tabular}
				\end{onlyenv}
				
			\end{center}
		\end{column}
	\end{columns}
	\note{
		Notes
	}
\end{frame}

\begin{frame}[fragile, t]{ARP-Spoofing Umsetzung}
	\begin{onlyenv}<1>
		\begin{block}{Probleme}
			\begin{itemize}
				\item fehlende Hilfsfunktionen
				\item Unterscheidung von Paketfeldern
				\item falsche filter-Implementierung
			\end{itemize}
		\end{block}
		
		\textbf{Datenpakete filtern} \\
		\begin{tabular}{c}
			\lstinputlisting[language=ADL, firstline=0, lastline=4, basicstyle=\scriptsize, firstnumber=1]{listings/arpServer.adl}
		\end{tabular}
	\end{onlyenv}

	\begin{onlyenv}<2>
		\begin{block}{Probleme}
			\begin{itemize}
				\item Vorinitialisierung
			\end{itemize}
		\end{block}
	
		\textbf{Daten auslesen} \\
		\begin{tabular}{c}
			\lstinputlisting[language=ADL, firstline=8, lastline=14, basicstyle=\scriptsize, firstnumber=8]{listings/arpServer.adl}
		\end{tabular}
	\end{onlyenv}


	\begin{onlyenv}<3>
		\begin{block}{Probleme}
			\begin{itemize}
				\item fehlende ARP-Implementierung
			\end{itemize}
		\end{block}
		
		\textbf{Neues ARP-Datenpaket erstellen} \\
		\begin{tabular}{c}
			\lstinputlisting[language=ADL, firstline=16, lastline=26, basicstyle=\scriptsize, firstnumber=16]{listings/arpServer.adl}
		\end{tabular}
	\end{onlyenv}

	\begin{onlyenv}<4>
		\begin{block}{Probleme}
			\begin{itemize}
				\item falsche Dokumentation
			\end{itemize}
		\end{block}
	
		\textbf{Letzte Änderungen} \\
		\begin{tabular}{c}
			\lstinputlisting[language=ADL, firstline=28, lastline=32, basicstyle=\scriptsize,firstnumber=28]{listings/arpServer.adl}
		\end{tabular}
	\end{onlyenv}
	\note{
		Notes
	}
\end{frame}

%\begin{frame}{Probleme bei der Umsetzung}
%	\begin{columns}
%		\begin{column}{0.5\textwidth}
%			\textbf{Fehlende...}
%			\begin{itemize}
%				\item Anbindung der Switches
%				\item ARP-Implementierung
%				\item Hilfsfunktionen
%			\end{itemize}
%		\end{column}
%		\begin{column}{0.5\textwidth}
%			\textbf{Falsche...}
%			\begin{itemize}
%				\item ADL-Definitionen
%				\item Dokumentation
%				\item Implementierung bedingter Angriffe
%			\end{itemize}
%		\end{column}
%	\end{columns}
%\end{frame}

\subsection{SYN-Flooding}

\begin{frame}[fragile]{SYN-Flooding}
	\begin{columns}
		\begin{column}{0.5\textwidth}
			\begin{center}
				\textbf{TCP-Handshake}
				\includestandalone[keepaspectratio]{tikz/baTCPHandshake}
			\end{center}
		\end{column}
		\begin{column}{0.5\textwidth}
			\begin{visibleenv}<1>
				\begin{center}
					\textbf{SYN-Flooding}
					\includestandalone[keepaspectratio]{tikz/baTCPSynFlood}
				\end{center}
			\end{visibleenv}
		\end{column}
	\end{columns}
	
	\note{
		Notes
	}
\end{frame}

\subsection{Firewall-Scanner}

\begin{frame}{ACK-Firewall-Scanner}
%	\visible<1>{\begin{block}{Implementierung}
%Implementierung ähnlich zu den vorherigen Angriff!
%	\end{block}}
	\begin{columns}
		\begin{column}{0.5\textwidth}
			\begin{center}
				\includestandalone[keepaspectratio]{tikz/baTCPAckAnswer}
			\end{center}
		\end{column}
		\begin{column}{0.5\textwidth}
			\begin{center}
				\textbf{Probleme}
				\begin{itemize}
					\item Wissen über Zustand des Systems
					\begin{itemize}
						\item Wurde das Gerät bereits getestet?
					\end{itemize}
					\item Firewall-Module
					\item Port-Range-Definieren
				\end{itemize}
			\end{center}
		\end{column}
	\end{columns}
	
	\note{
		Notes
	}
\end{frame}

%\begin{frame}{SEA++-Änderungen}
%	\begin{columns}[onlytextwidth,t] 
%		\begin{column}{0.5\textwidth}
%			\textbf{Änderungen am Quellcode}
%			\begin{enumerate}
%				\item<1-> Fehlerbehebung
%				\item<2-> Erweiterung um ARP
%				\begin{itemize}
%					\item Erweiterbarkeit beschrieben im Handbuch
%				\end{itemize}
%			\end{enumerate}
%		\end{column}
%		\begin{column}{0.5\textwidth}
%			\begin{onlyenv}<1>
%				\textbf{\adl{filter} $\Rightarrow$ Absturz}
%				\begin{itemize}
%					\item jedes Datenpaket betrachten
%					\item Schicht prüfen 
%					\item wenn Datenpaketklasse unbekannt 
%					\begin{itemize}
%						\item[$\Rightarrow$] Unbehebbarer Fehler
%					\end{itemize}
%				\end{itemize}
%			\end{onlyenv}
%			\begin{onlyenv}<2->
%				\textbf{\only<2>{3}\only<3>{4} Erweiterungen}
%				\begin{enumerate}
%					\item \file{seapputils.cc} $\Rightarrow$ Schicht zurückgeben
%					\item \file{Create.h} $\Rightarrow$ Neues Enum
%					\item \file{Create.cc} $\Rightarrow$ Paket zusammenbauen
%					\item<3> \file{Change.cc} $\Rightarrow$ Typecast
%				\end{enumerate}
%			\end{onlyenv}
%		\end{column}
%	\end{columns}
%\end{frame}
