\begin{frame}{Thema}
	\begin{block}{Titel}
		Simulative Analyse von Cyber-Angriffen am Beispiel des Netzes der Fakultät für Informatik der TU Dortmund
	\end{block}
	\begin{columns}
		\begin{column}{0.5\textwidth}
			\begin{center}
				Bild aus Urheberrechtsgründen entfernt
			\end{center}
		\end{column}
		\begin{column}{0.5\textwidth}
			\begin{visibleenv}<2>
				\begin{itemize}
					\item Drei Angriffe
					\begin{itemize}
						\item ARP-Spoofing
						\item SYN-Flooding
						\item Port-Scanner
					\end{itemize}
					\item Fokus auf Angriffssimulator
					\item Weitere Einschränkungen
					\begin{itemize}
						\item Nicht Programmieren
						\item Arbeit im direkten Programmökosystem
					\end{itemize}
				\end{itemize}
			\end{visibleenv}
		\end{column}
	\end{columns}
	
	\note{
		\begin{itemize}
			\item Simulative Analyse von Cyber-Angriffen am Beispiel des Netzes der Fakultät für Informatik der TU Dortmund
			\item Titel sehr Allgemein
			\item Deswegen Einschränkungen
			\item Aus ersten Erkenntnissen Fokus auf SEA gelegt
			\item Drei Beispielhafte Angriffe, welche möglichst verschieden sind
			\item Weitere Einschränkungen waren Nötig
			\begin{itemize}
				\item Es werden keine eigenen Module für omnet erstellt und es werden nur dierekte OMNET Programme genutzt. Weitere zu dem Ökosystem kompatible Programme wie Python, R werden nicht betrachtet!
			\end{itemize}
		\end{itemize}
	}
\end{frame}