
\newglossaryentry{seapp}{name={SEA++},description={SEA++ ist ein Netzwerkangriffssimulator für OMNeT++. Er erlaubt die Analyse erfolgreicher Angriffe auf ein Computernetzwerk}}
\newglossaryentry{omnetpp}{name={OMNeT++},description={Ein Discrete-Event-Simulator welcher hauptsächlich für Netzwerksimulationen genutzt wird}}
\newglossaryentry{inet}{name={INET},description={Erweiterung für OMNeT++}}
\newglossaryentry{ned}{name={NED},description={Eine High-Level-Language in der eine (Neztwerk)-Topologie beschrieben werden kann. Steht für Network Description.}}
\newglossaryentry{c++}{name={C++},description={Eine genormte Programmiersprache}}
\newglossaryentry{ethernet}{name={Ethernet},description={Schicht 1 und 2 Übertragungstandard. Standarisiert in IEEE 802.3}}
\newglossaryentry{gnur}{name={GNU R},description={Umgebebung zur Berechnung und Darstellung von Statistiken \cite{GnuRWebsite}}}
\newglossaryentry{python}{name={Python},description={Open Source Porgrammiersprache}}
\newglossaryentry{sterntopologie}{name={Sterntopologie}, plural={Sterntopologien}, description={Eine Sterntopologie verbindet alle Geräte von einem zentralen Knoten aus \cite{Kurose02}}}
\newglossaryentry{git}{name={Git}, description={Eine bekannte verteilte Versionierungssoftware \cite{GitWebsite}}}
\newglossaryentry{svn}{name={SVN}, description={Eine bekannte zentralisierte Versionierungssoftware \cite{SvnWebsite}}}

\newglossaryentry{tcpip}{name={Internet-Protokoll-Stack},description={Der Internet-Protokoll-Stack ist eine Einteilung verschiedner Protokolle in Schichten. Vorgestellt in \autocite{Frystyk1994}. Meist als fünfstelliges Modell wie in \citetitle{Kurose02} \cite{Kurose02} dargestellt}}

\newglossaryentry{layer5}{name={Anwendungsschicht}, description={Oberste Schicht im Internet Protokoll Stack. Die deutsche Bezeichnung ist entnommen aus dem Standardwerk Computernetzwerke \cite{Kurose02}}}
\newglossaryentry{layer4}{name={Transportschicht}, description={Vierte Schicht im Internet Protokoll Stack \cite{Kurose02}}}
\newglossaryentry{layer3}{name={Vermittlungsschicht}, description={Dritte Schicht im Internet Protokoll Stack \cite{Kurose02}}}
\newglossaryentry{layer2}{name={Sicherungsschicht}, description={Zweite Schicht im Internet Protokoll Stack \cite{Kurose02}}}
\newglossaryentry{layer1}{name={Bitübertragungsschicht}, description={Unterste Schicht im Internet Protokoll Stack \cite{Kurose02}}}

\newglossaryentry{ieee8021q}{name={IEEE802.1Q}, description={Standard in dem Grundlagen für VLAN festgeschrieben wurden \cite{8403927}}}

\newglossaryentry{bbbg}{name={BBB},	description={Eine Onlinevideokonferenz Software speziell für die Lehre \cite{BBBWebsite}}}
\newglossaryentry{bbb}{type=\acronymtype, name={BBB}, description={BigBlueButton}, first={BigBlueButton (BBB)\glsadd{bbbg}}, see=[Glossary:]{bbbg}}

\newacronym{igmp}{IGMP}{Internet Group Management Protocol}
\newacronym{icmp}{ICMP}{Internet Control Message Protocol}
\newacronym{ipv4}{IPv4}{Internet Protocol Version 4}
\newacronym{bsi}{BSI}{Bundesamt für Sicherheit in der Informationstechnik}
\newacronym{neta}{NETA}{NETwork Attacks}
\newacronym{ppp}{PPP}{Point-to-Point Protocol}
\newacronym{stp}{STP}{Spanning Tree Protocol}
\newacronym{mitm}{MitM}{Man-in-the-Middle-Angriff}
\newacronym{arp}{ARP}{Address Resolution Protocol}
\newacronym{rfc}{RFC}{Request for Comments}
\newacronym{vss}{VSS}{Virtual Switching System}
\newacronym{vlan}{VLAN}{Virtual Local Area Network}
\newacronym{itmc}{ITMC}{IT \& Medien Centrum}
\newacronym{tu}{TU}{Technische Universität Dortmund}
\newacronym{dnf}{DFN}{Deutsche Forschungsnetz}
\newacronym{irb}{IRB}{Informatikrechner-Betriebsgruppe}
\newacronym{wlan}{WLAN}{Wireless Local Area Network}
\newacronym{adl}{ADL}{Attack Description Language}
\newacronym{asl}{ASL}{Attack Specification Language}
\newacronym{asi}{ASI}{Attack Specification Interpreter}
\newacronym{ase}{ASE}{Attack Specification Engine}
\newacronym{gep}{GEP}{Global Event Processor}
\newacronym{lep}{LEP}{Local Event Processor}
\newacronym{dos}{DoS}{Denial of Service}
\newacronym{dosa}{DoS-Angriff}{Denial-of-Service-Angriff}
\newacronym{ddos}{DDoS}{Distributed Denial of Service}
\newacronym{ddosa}{DDoS}{Distributed-Denial-of-Service-Angriff}
\newacronym{ide}{IDE}{Integrierte Entwicklungsumgebung}
\newacronym{mac}{MAC}{Media Access Control}
\newacronym{csmacd}{CSMA/CD}{Carrier Sense Multiple Access/Collision Detection}
\newacronym{nic}{NIC}{Network Interface Card}
\newacronym{rip}{RIP}{Routing Information Protocol}
\newacronym{bgp}{BGP}{Border Gateway Protocol}
\newacronym{xml}{XML}{Extensible Markup Language}
\newacronym{tcp}{TCP}{Transmission Control Protocol}
\newacronym{udp}{UDP}{User Datagram Protocol}
\newacronym{dsl}{DSL}{domänenspezifische Sprache}
\newacronym{ftp}{FTP}{File Transfer Protocol}
\newacronym{http}{HTTP}{Hypertext Transfer Protocol}
\newacronym{ssh}{SSH}{Secure Shell}
\newacronym{oh}{OH}{Otto-Hahn-Straße}
\newacronym{tucsnet}{Informatiknetzwerk}{Netzwerk der Fakultät Informatik an der TU-Dortmund}
\newacronym{rtt}{RTT}{Round Trip Time}
\newacronym{tls}{TLS}{Transport Layer Security}
\newacronym{pdu}{PDU}{Protocol Data Unit}

\glsaddall
