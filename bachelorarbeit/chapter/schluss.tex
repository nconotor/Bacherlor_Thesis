\chapter{Zusammenfassung}\label{chap:schluss6}
\gls{seapp} wurde von dessen Autoren als Möglichkeit der Cyberangriffanalyse im \gls{omnetpp} Ökosystem vorgestellt. Dies wurde anhand eines \gls{dos}-SYN-Flooding-Angriffs, eines ARP-Poisoning-Angriffs und eines Port-Scanners untersucht. In jedem der Teilschritte zum Erstellen des Modells und der Simulation haben sich Probleme und Vorteile von \gls{seapp} gezeigt.

Bei dem ersten Schritt der Implementierung einer Netzwerktopologie kamen Probleme bei der Nutzung des Programmökosystems auf. Wie in Absatz \ref{aus:Modularität} beschrieben, konnte eine anfänglich geplante Modularität des Netzwerks nicht erreicht werden. Dies führte zu einer Einschränkung in der Darstellung des Netzwerks und zusätzlichem Implementierungsaufwand.

Ein weiteres Problem bei der Implementierung waren Einschränkungen durch den Umfang von \gls{inet} in der alten Version 2.6, auf welche in Absatz \ref{aus:alteVersionen} eingegangen wurde. Hier zeigten sich fehlende Umsetzungen von Kerntechnologien des Informatiknetzwerks, welche eine Abweichung von der realen Topologie verlangten. Trotz dieser Probleme konnte eine vereinfachte Version des Netzwerks der Fakultät für Informatik an der \gls{tu}, mit geringer Einarbeitungszeit, implementiert werden. 

Im nächsten Schritt, der Konfiguration des Netzwerks, sind weitere Probleme aufgekommen. Vor allem fehlende Möglichkeiten im Bereich der Anwendungen haben die Aussagekraft des Modells eingeschränkt. Absatz \ref{aus:fehlendeModule} führt insbesondere Module für Netzwerkdatenverkehr und Sicherheitsfeatures als Beispiele auf. Trotz dieser Einschränkungen konnte ein grobes Modell von dem Netzwerkdatenverkehr erstellt werden, welches wenig Flexibilität zeigte.

Nachdem die Topologie erstellt wurde, zeigten sich Probleme bei der Einarbeitung in die \gls{seapp} Dokumentation, welche in Absatz \ref{aus:Dokumentation} ausgeführt werden. In einigen Bereichen, wie zum Beispiel der Benennung der vorgestellten \gls{dsl} oder der Notation der Felder in einem Filter, wichen die jeweiligen Veröffentlichungen voneinander ab. Insbesondere  die \adl{drop}-Anweisung war nicht in der vorgestellten Syntax nutzbar. Durch Sichtung der Implementierung konnten diese Probleme umgangen werden und die jeweiligen Angriffe in \gls{adl}-Dateien mit der \gls{asl} kurz und präzise implementiert werden.

Es sei angemerkt, dass bei der Umsetzung des Port-Scanners, welcher einen bedingten Angriff benutzte, und bei der Erweiterung von \gls{seapp} um ARP-Datenpakete, erneut Probleme aufgetreten sind. Diese werden in Abschnitt \ref{aus:Erweiterbarkeit} besprochen. Im Fall der ARP-Erweiterung fehlte es an Dokumentation für Adressfelder des Paketes, welche speziell behandelt werden müssen. Ein anderes Problem zeigte sich bei unbekannten Datenpaketen, welche durch eine Filterbedingung behandelt werden und zu einem Absturz führen. Dieses Verhalten sollte in der ausgelieferten Konfiguration nicht vorhanden sein.

Die Implementierung wurde eingeschränkt durch fehlende Sprachfeatures, welche in Abschnitt \ref{aus:Sprachumfang} besprochen wurden. Im Besonderen sind hier Konstrukte zum Erstellen mehrere Datenpakete und Konditionalabfragen zu nennen. Solche Funktionalitäten würden den Umgang mit \gls{seapp} erleichtern, waren aber bei der Definition der Angriffe nicht essenziell.

Im weiteren Verlauf zeigten sich weitere Probleme mit den Implementierungen. Die Benutzung der Knoten Id, welche von \gls{omnetpp} als Identifikator des Knotens zugewiesen wird, erwies sich als eine problematische Abhängigkeit zur Topologie des Netzes. In Abschnitt \ref{aus:Flexibilität} werden die daraus resultierenden Nachteile aufgezeigt. Hervorzuheben ist vor allem die fehlende Anpassbarkeit der Netzwerktopologie, bei der kleine Änderungen die Knoten Ids verschieben können. 

Abschließend wurden die Ergebnisse, welche aus dem Modell gewonnen werden konnten, besprochen. Bei der Analyse sind weitere Probleme aufgetreten. Da nur die Analysemöglichkeiten vom \gls{omnetpp} genutzten worden sind, waren feine Filterungen der Daten nicht möglich. Im Abschnitt \ref{aus:Analyse} werden die Konsequenzen davon aufgeführt. Es sei insbesondere darauf hingewiesen, dass neuere \gls{omnetpp} Versionen dieses Problem angehen.

Zuletzt wurde bei der Analyse ein ungeklärtes Problem erkannt, welches Thema des Abschnitts \ref{aus:Zufallszahlen} ist. Obwohl die Zeit und der Seed der Simulation fest auf ein Datum eingeschränkt sind, geben technisch gleiche Durchläufe unterschiedliche Ergebnisse aus. Weitere Untersuchungen sind nötig, um dieses Problem abschließend zu klären.

Keines der Probleme hat die Analyse der Daten verhindert. Hauptsächlich wurde sich bei der Analyse auf den Netzwerkdurchsatz sowie die Summe der gesendeten Pakete konzentriert. Dabei konnten für alle Angriffe Ergebnisse gewonnen werden.

Die Effekte des ARP-Spoofing waren wie vermutet deutlich im Netzwerk zu erkennen. Je nach angegriffenem Bereich kann ein starker Rückgang im Netzwerkverkehr gemessen werden. Insbesondere bei dem Angriff auf den Serverbereich, ist ein Großteil der Netzwerkaktivität weggefallen. Weiterhin kann festgestellt werden, dass der Netzwerkverkehr in beide Richtungen einbricht.

Bei dem \gls{dos}-Angriff konnte kein kompletter Netzwerkausfall festgestellt werden. Obwohl die erhöhte Paketrate ab Angriffsbeginn am Router zu beobachten war, konnte kein signifikanter Rückgang im Netzwerkvekehr der Lehrstühle festgestellt werden. Aufgrund der unklareren Situation im Vergleich zum ARP-Angriff wurden der Quality-of-Service-Parameter RTT und der Pingverlust betrachtet, wobei über ersteren keine genaue Aussage ohne genauere Analyse mit erweiterten Programmen getroffen werden konnte. Ein erhöhter Pingverlust konnte beobachtet werden, sollte aber aufgrund der Problematik mit den Zufallsvariablen genauer betrachtet werden.

Der Port-Scanner hat die geringsten Auswirkungen gezeigt. Wieder wurde der Netzwerkverkehr untersucht, bei dem erneut wegen dem Problem mit den Zufallsvariablen kein eindeutiger Trend zu erkennen war. Zusätzliche Parameter des TCP-Moduls konnten nicht untersucht werden, sodass kein eindeutiges Fazit gezogen werden konnte.

Zusammenfassend kann gesagt werden, dass die Analyse mit \gls{seapp} insbesondere für Forschergruppen mit bestehenden \gls{omnetpp}-Modellen interessant ist, welche auch entsprechend weitreichende Abläufe für eine Analyse besitzen und eigene Module für aufgekommene Probleme dieser Bachelorarbeit besitzen, wie \zB Netzwerkdatenverkehrsgeneratoren. 

Insbesondere Probleme wie Abstürze durch unbekannte Pakete in der Standardkonfiguration, der eingeschränkte Sprachumfang, die Dokumentation und Probleme bei der Modularität der Topologien schränken \gls{seapp} im Moment ein und sollten behoben werden, damit mehr Beiträge zum \gls{seapp}-Ökosystem entstehen. 