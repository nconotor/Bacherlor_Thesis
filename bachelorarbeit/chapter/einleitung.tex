\chapter{Einleitung}\label{chap:einleitung1}
Im Frühjahr des Jahres 2020 musste wegen eines Cyber-Angriffs auf die Ruhr-Universität-Bochum \cite{RUBHH} ein Großteil der internen IT-Infrastruktur von dem Netz genommen werden. Dies ist nur ein Beispiel für eine Vielzahl von Angriffen auf die deutsche IT-Infrastruktur. Das \gls{bsi} sammelte im Zeitraum des Lageberichts 2019 bis zu 11,5 Millionen Schadprogramminfektionsmeldungen von deutschen Netzbetreibern. Es wurden weiterhin tägliche {ca.} 110.000 Botinfektion auf deutsche Netze gemeldet und 770.000 Mails an das Regierungsnetz mit Schadprogrammen abgefangen \cite[\seite{34-35}]{BSILage}.

Bei diesen Zahlen handelt es sich nur um die erkannten oder vereitelten Angriffe auf Netzwerke. Es ist zu vermuten, dass es eine hohe Dunkelziffer \cite{BKA19} an unerkannten Angriffen gibt. Einige großangelegte staatliche Angriffe, gegen Ziele wie Atomkraftwerke, sind jahrelang nicht bemerkt \cite[\kapitel{9}]{Monte2015} worden. Dies zeigt ein großes Bedrohungspotential von nicht verhinderbaren Cyber-Angriffen und stellt die Frage nach den Auswirkungen dieser Angriffe.

\section{Lösungen}
Um diese Frage zu beantworten, müssen Cyber-Angriffe untersucht werden, wofür es verschiedene Herangehensweisen gibt. Reale Netzwerke können angegriffen werden und dabei analysiert werden, um etwaige Folgen weiterer Angriffe zu minimieren. Weiterhin sind Netzwerksimulationen möglich, um das gleiche Ziel ohne Nutzung eines realen Netzwerks zu realisieren.

Simulative Ansätze haben die Abstrahierung vom realen Fall als Vorteil, wodurch Modellanpassungen die Untersuchung einer Vielzahl von Szenarien erlauben. Ein Nachteil ist die Aussagekraft, welche von der Modellierung abhängt. In dieser Arbeit soll der simulierte Ansatz gewählt werden. Für die Simulation von Netzwerken gibt es eine Vielzahl von möglichen Programmen \cite{SimOverview}, wie NS2, NS3, \gls{omnetpp}, SSFNet und J-Sim. In dieser Bachelorarbeit wird der Event-Simulator \gls{omnetpp} genutzt. Auf \gls{omnetpp} bauen eine Vielzahl von Erweiterungen\footnote{Eine Übersicht von Erweiterungen kann auf der \href{https://omnetpp.org/download/models-and-tools}{Projektseite} von \gls{omnetpp} gefunden werden.} auf. 

Für die Analyse von Auswirkungen einer Cyber-Attacke wurde die Erweiterung \gls{seapp} entwickelt, welche auf \gls{inet} aufbaut. \gls{seapp} wurde in \citetitle{Tiloca2019} als flexible und nutzerfreundliche Möglichkeit Cyber-Angriffe zu analysieren, vorgestellt  \cite[\seite{276}]{Tiloca2019}. Zuerst besprochen wurde SEA++ im Jahr 2014 in einer wissenschaftlichen Arbeit \cite{Tiloca2014}. Für \gls{seapp} existiert zusätzlich ein Benutzerhandbuch \cite{SEAManual}. \gls{seapp} ist nicht die einzige Erweiterung, welche Sicherheitsthematiken im \gls{omnetpp} Ökosystem behandelt. Ein Beispiel für eine alternative Erweiterung ist \gls{neta} \cite{NETA}.

\section{Herangehensweise}
In dieser Arbeit sollen \gls{seapp} und \gls{omnetpp} genutzt werden, um verschiedene Cyberangriffe auf ein Universitätsnetzwerk auszuführen. Der Fokus liegt dabei auf der Arbeit mit \gls{seapp}, welche am Ende besprochen wird. Dazu werden die einzelnen Aspekte von \gls{seapp} vorgestellt und das Programmökosystem betrachtet.

Die Wahl der Cyberangriffe spiegelt typische Teileaspekte eines größeren Cyberangriffs wieder. Es soll ein \gls{dosa} untersucht werden, welcher zu den häufigsten Angriffsarten im Internet gehört. Für die Betrachtung eines Angriffs, welcher von einem Gerät innerhalb eines Netzwerks ausgeht, wird eine Variante des ARP-Poisonings implementiert. Auf ARP-Poisoning bauen viele andere unterschiedliche Angriffe auf, wie \zB ein \gls{mitm}. Zuletzt wird ein Port-Scanner betrachtet, welcher unabdingbar ist, um Informationen über ein Netzwerk zu erlangen.

Aufgrund der vermuteten unterschiedlichen Auswirkungen eignen sich die Angriffe zusätzlich für die Betrachtung von \gls{seapp}. Es wird vermutet, dass die Auswirkungen von einem ARP-Poisoning und dem \gls{dosa} deutlich im Netzwerk zu sehen sind, der Port-Scanner hingegen eher unauffällig ist.

\section{Einteilung}
Diese Arbeit ist in sechs Teile gegliedert. Nach der Einleitung folgt eine Vorstellung des Programmökosystems \ref{chap:programme2}, mit dem gearbeitet wird. Vor allem die Syntax und Bedienung von \gls{omnetpp} und \gls{seapp} sollen dabei anhand einfacher Beispiele ergründet werden. In Teil \ref{chap:methodik3} folgt eine Vorstellung der Angriffe und eine Übersicht über das reale Netzwerk der Informatikfakultät an der \gls{tu}. Teil \ref{chap:sim4} stellt die Modellierung des realen Netzwerks in \gls{omnetpp} vor und zeigt die Implementierung der Angriffe mit \gls{seapp}. Insbesondere wird auf die Unterschiede zwischen Modell und Realität eingegangen. Im vorletzten Teil \ref{chap:auswertung5} werden die Ergebnisse der Angriffe besprochen sowie die Probleme und Vorteile von \gls{seapp} aufgezeigt. In der Zusammenfassung \ref{chap:schluss6} werden die gewonnen Ergebnisse zusammengefasst. Rohdaten, wie Grafiken und Dateien, welche in den einzelnen Kapiteln nicht vollständig dargestellt werden können, befinden sich im Anhang \ref{app:start} bis \ref{app:ende}.