\gls{seapp} wurde für eine ältere Version von \gls{omnetpp} entwickelt, weshalb die Installationsanleitung nicht mehr zu einer problemlosen Installation der Anwendung führt. Um die Reproduzierbarkeit der Ergebnisse zu sichern, sind im Folgenden die für diese Arbeit genutzten Installationsschritte aufgeführt. Alle Schritte wurden auf einen Ubuntu 20.04 und 18.04 Durchgeführt. 
\begin{enumerate}
	\item Zuerst muss \href{https://omnetpp.org/download/old}{\gls{omnetpp} 4.6} Installiert werden. Die in der aktuellen Installationsanleitung\footnote{\href{https://doc.omnetpp.org/omnetpp/InstallGuide.pdf}{Genutzte Installationsanleitung}} funktionieren für die älteren \gls{omnetpp} Versionen. 
	\item Als nächstes muss manuell \href{https://inet.omnetpp.org/Download.html}{\gls{inet}} in Version 2.6 installiert\footnote{Es kann ein Compileerror auftreten. Die in der \href{https://groups.google.com/forum/\#!topic/omnetpp/96orSuS3Whg}{Google Group} vorgeschlagene Lösung kann auch für \gls{inet} 2.6 verwendet werden} werden. Der automatische Download von \gls{inet} durch \gls{omnetpp} schlägt bei der Version fehl.
	\item Nach der erfolgreichen Installation von \gls{inet} muss \href{https://omnetpp.org/download-items/SEA++.html}{\gls{seapp}} installiert werden. Eine Anleitung hierfür wird von \gls{seapp}\cite{SEAManual} bereitgestellt.
\end{enumerate}